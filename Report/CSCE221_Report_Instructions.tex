%% LyX 2.3.3 created this file.  For more info, see http://www.lyx.org/.
%% Do not edit unless you really know what you are doing.
\documentclass[english]{article}
\usepackage[latin9]{inputenc}
\usepackage{geometry}
\geometry{verbose,tmargin=2.5cm,bmargin=2.5cm,lmargin=2.5cm,rmargin=2.5cm,headheight=0cm,headsep=0cm}
\usepackage{babel}
\usepackage[unicode=true]
 {hyperref}

\makeatletter

%%%%%%%%%%%%%%%%%%%%%%%%%%%%%% LyX specific LaTeX commands.
%% Because html converters don't know tabularnewline
\providecommand{\tabularnewline}{\\}

\@ifundefined{date}{}{\date{}}
\makeatother

\begin{document}
\title{CSCE 221 - Programming Assignment 4 Report \textbf{(20 points)}}
\maketitle
\begin{center}
{\large{}Due April 14, 2021}{\large\par}
\par\end{center}
\author{First Name:~~~~~~~~~~~~~~~~~Last Name: ~~~~~~~~~~~~~~~~~~~~~~UIN:~~~~~~~~~~~~~~}
\author{User Name: ~~~~~~~~~~~~~~~~~~~~~~E-mail address:~~~~~~~~~~~~~~~~~~~~~~~~~~~~~~~~\medskip{}
}
\begin{quotation}
Please list all sources in the table below including web pages which
you used to solve or implement the current homework. If you fail to
cite sources you can get a lower number of points or even zero, read
more in the Aggie Honor System Office \href{http://aggiehonor.tamu.edu/}{http://aggiehonor.tamu.edu/}\medskip{}
\medskip{}
\end{quotation}
\begin{tabular}{|c|c|c|c|c|c|}
\hline 
\textbf{Type of sources} & ~~~~~~~~~~~~~~~~~~ & ~~~~~~~~~~~~~~~~~~ & ~~~~~~~~~~~~~~~~~~ & ~~~~~~~~~~~~~~~~~~ & ~~~~~~~~~~~~~~~~~~\tabularnewline
\hline 
People &  &  &  &  & \tabularnewline
\hline 
Web pages (provide URL) &  &  &  &  & \tabularnewline
\hline 
Printed material &  &  &  &  & \tabularnewline
\hline 
Other Sources &  &  &  &  & \tabularnewline
\hline 
\end{tabular}
\date{\medskip{}
\medskip{}
}
\begin{quotation}
I certify that I have listed all the sources that I used to develop
the solutions/code to the submitted work.

\textquotedblleft \emph{On my honor as an Aggie, I have neither given
nor received any unauthorized help on this academic work.}\textquotedblright{} 
\end{quotation}
\date{\medskip{}
\medskip{}
}

\begin{tabular}{ccccc}
Your Name (signature) & ~~~~~~~~~~~~~~~~~~~~~~~~~~~ & ~~~~~~~~~~~~~~~~~~~~~ & Date & ~~~~~~~~~~~~~~~~~~~~\tabularnewline
\end{tabular}\pagebreak{}
\begin{enumerate}
\item The description of the assignment problem.\vfill{}
\item The description of data structures and algorithms used to solve the
problem.
\begin{enumerate}
\item Provide definitions of data structures by using Abstract Data Types
(ADTs) \vfill{}
\item Write about the ADTs implementation in C++ (for all the three MPQs).\vfill{}
\item Describe algorithms used to solve the problem. For every MPQ\texttt{
(UnsortedMPQ}, \texttt{SortedMPQ} and \texttt{BinaryHeapMPQ}), list
the MPQ functions (\texttt{remove\_min(), is\_empty(), min(),} and
\texttt{insert()}) and provide their descriptions.\vfill{}
\item Show the time complexity analysis for the following. Time complexity
analysis means providing a \textbf{basic runtime function/recurrence
relation}, \textbf{solution for recurrence relation with steps (wherever
needed)} and a \textbf{Big-O} Notation:
\begin{enumerate}
\item \textbf{Best, worst, and average case} of each of the MPQ functions
(\texttt{\textbf{remove\_min(), is\_empty(), min(),}}\textbf{ and
}\texttt{\textbf{insert()}}) for \texttt{\textbf{UnsortedMPQ.}} (Note:
Some functions may have same runtimes for all the cases. In that case,
write the answer only once and mention that the runtime applies to
all the cases.).\vfill{}

\begin{enumerate}
\item Provide an \textbf{example for} \textbf{best, worst, and average case}
for \texttt{\textbf{UnsortedMPQ}}. \vfill{}
\end{enumerate}
\item \textbf{Best, worst, and average case} of each of the MPQ functions
(\texttt{\textbf{remove\_min(), is\_empty(), min(),}}\textbf{ and
}\texttt{\textbf{insert()}}) for \texttt{\textbf{SortedMPQ.}} (Note:
Some functions may have same runtimes for all the cases. In that case,
write the answer only once and mention that the runtime applies to
all the cases).\vfill{}

\begin{enumerate}
\item Provide an \textbf{example for} \textbf{best, worst, and average case}
for \texttt{\textbf{SortedMPQ}}. \vfill{}
\end{enumerate}
\item \textbf{Best, worst, and average case} of each of the MPQ functions
(\texttt{\textbf{remove\_min(), is\_empty(), min(),}}\textbf{ and
}\texttt{\textbf{insert()}}) for \texttt{\textbf{BinaryHeapMPQ.}}
(Note: Some functions may have same runtimes for all the cases. In
that case, write the answer only once and mention that the runtime
applies to all the cases).\vfill{}

\begin{enumerate}
\item Provide an \textbf{example for} \textbf{best, worst, and average case}
for \texttt{\textbf{BinaryHeapMPQ}}. \vfill{}
\pagebreak{}
\end{enumerate}
\end{enumerate}
\end{enumerate}
\item A C++ organization and implementation of the problem solution 
\begin{enumerate}
\item Provide a list and description of classes or interfaces used by a
program such as classes used to implement the data structures or exceptions.\vfill{}
\item Provide features of the C++ programming paradigms like Inheritance
or Polymorphism in case of object oriented programming, or Templates
in the case of generic programming used in your implementation. \vfill{}
\end{enumerate}
\item A user guide description how to navigate your program with the instructions
how to: 
\begin{enumerate}
\item compile the program: specify the directory and file names, etc.\vfill{}
\item run the program: specify the name of an executable file. \vfill{}
\pagebreak{}
\end{enumerate}
\item Specifications and description of input and output formats and files 
\begin{enumerate}
\item The type of files: keyboard, text files, etc (if applicable). \vfill{}
\item A file input format: when a program requires a sequence of input items,
specify the number of items per line or a line termination. Provide
a sample of a required input format. \vfill{}
\item Discuss possible cases when your program could crash because of incorrect
input (a wrong file name, strings instead of a number, or such cases
when the program expects 10 items to read and it finds only 9.)\vfill{}
\end{enumerate}
\item Provide types of exceptions and their purpose in your program (Answer
only to the ones that are applicable for this assignment).
\begin{enumerate}
\item logical exceptions (such as deletion of an item from an empty container,
etc.).\vfill{}
\item runtime exception (such as division by $0$, etc.)\vfill{}
\end{enumerate}
\item Include evidence of your testing by providing screenshots. Screenshots
should show execution of the 5 main methods \texttt{(unsortedmpq-main.cpp,
sortedmpq-main.cpp, main.cpp, cpu-job-main.cpp, binaryheap-mpq-main.cpp).}\vfill{}
\item Provide graphs and data tables of your CPU timing simulation results.
Graph should be plotted for \textbf{runtime vs. input size}. The input
sizes are 4, 10, 100, and 1,000. To obtain this data, compile and
run \texttt{main.cpp}. Choose option ``2. Timing Simulation''. Provide
the input file name (SetSize4.txt) and output filename. After execution,
you will find the output file in ``OutputFiles'' folder. The timing
for all the three MPQ implementations will be displayed. Fill it in
the following table and plot it as a \textbf{graph}.

\begin{tabular}{|c|c|c|c|}
\hline 
 & \multicolumn{3}{c|}{Runtime}\tabularnewline
\hline 
Input Sizes & Unsorted MPQ & Sorted MPQ & Binary heap MPQ\tabularnewline
\hline 
4 (SetSize4.txt) &  &  & \tabularnewline
\hline 
10 (SetSize10.txt) &  &  & \tabularnewline
\hline 
100 (SetSize100.txt) &  &  & \tabularnewline
\hline 
1000 (SetSize1000.txt) &  &  & \tabularnewline
\hline 
\end{tabular}
\end{enumerate}

\end{document}
